\documentclass[12pt, a4paper]{article}
\usepackage{caption}
\usepackage{graphicx}
\usepackage{hyperref}
\hypersetup{
    colorlinks,
    citecolor=black,
    filecolor=black,
    linkcolor=black,
    urlcolor=black
}
\usepackage{tikz-network}
\usepackage{amsmath, amsfonts, amssymb, amsthm}
\usepackage{algpseudocode}
\usepackage{algorithm}
\title{Network and Cybersecurity\\ Exercises}
\date{2022}
\author{Kristoffer Klokker}

\usepackage{xcolor,listings}
\usepackage{textcomp}
\usepackage{color}
\usepackage{listings}
\definecolor{codegreen}{rgb}{0,0.6,0}
\definecolor{codegray}{rgb}{0.5,0.5,0.5}
\definecolor{codepurple}{HTML}{C42043}
\definecolor{backcolour}{HTML}{F2F2F2}
\definecolor{bookColor}{cmyk}{0,0,0,0.90}  
\color{bookColor}

\lstset{upquote=true}

\lstdefinestyle{mystyle}{
    backgroundcolor=\color{backcolour},   
    commentstyle=\color{codegreen},
    keywordstyle=\color{codepurple},
    numberstyle=\numberstyle,
    stringstyle=\color{codepurple},
    basicstyle=\footnotesize\ttfamily,
    breakatwhitespace=false,
    breaklines=true,
    captionpos=b,
    keepspaces=true,
    numbers=left,
    numbersep=10pt,
    showspaces=false,
    showstringspaces=false,
    showtabs=false,
    tabsize=3,
}
\lstset{style=mystyle}
\usepackage{zref-base}

\makeatletter
\newcounter{mylstlisting}
\newcounter{mylstlines}
\lst@AddToHook{PreSet}{%
  \stepcounter{mylstlisting}%
  \ifnum\mylstlines=1\relax
    \lstset{numbers=none}
  \else
    \lstset{numbers=left}
  \fi
  \setcounter{mylstlines}{0}%
}
\lst@AddToHook{EveryPar}{%
  \stepcounter{mylstlines}%
}
\lst@AddToHook{ExitVars}{%
  \begingroup
    \zref@wrapper@immediate{%
      \zref@setcurrent{default}{\the\value{mylstlines}}%
      \zref@labelbyprops{mylstlines\the\value{mylstlisting}}{default}%
    }%
  \endgroup
}

% \mylstlines print number of lines inside listing caption
\newcommand*{\mylstlines}{%
  \zref@extractdefault{mylstlines\the\value{mylstlisting}}{default}{0}%
}
\makeatother


\newcommand\numberstyle[1]{%
    \footnotesize
    \color{codegray}%
    \ttfamily
    \ifnum#1<10 0\fi#1 |%
}


\begin{document}
	\maketitle
	\clearpage
	\tableofcontents
	\clearpage
	\section{Lecture 1}
		\subsection{Generalize a formula for sending P such packets back-to-back over the N link}
			For a single package the sending time will be:
			$$d_{end-to-end}=N\frac{L}{R}$$
			For a number of packages $P$ the formular wil lbe:
			$$d_{end-to-end}=(P+N)\frac{L}{R}$$
		\subsection{Consider the circuit-switched network in this figure}
			\begin{figure}[h!]
				\includegraphics[width=300px]{assets/1.1.png}
				\center
			\end{figure}
			\subsubsection{What is the maximum number of simultaneous conections that can be in progress at the same time in this network}
				Assuming each host can have 4 ingoing and 4 outgoing connections the total number of active connections would be 12. This is where 4 simultaneous connections between the neightboor host.\\
				In case each computer can maximux have 4 ingoing or outgoing connections the total connections would be halfed to 6.
			\subsubsection{Suppose that all connections are between switches A and C. What is the maximum number of simultaneous conections that can be in progress}
				There can only be 4 connections between A and C.
			\subsubsection{Suppose we want to make four connections between switches A and C, and another four connections between B and D. Can we route these calls through the four links to accomodate all eight connections}
				To make a connection one pair can have the two outer links and the other pair can have the two inner links, resulting in 4 links between the pairs.
		\subsection{This elementary problem begins to explore propagation delay and transmission delay, two central concepts in data networking. Consider two hosts, A and B, connected by a single link of rate R bps. Suppose that the two hosts are separated by $m$ meters, and suppose the propagation speed along the link is $s$ meters/sec. Host A is to send a packet of size $L$ bits to Host B}
			\subsubsection{Express the propagation delay, $d_{prop}$ in terms of $m$ and s.}
				
				$$d_{prop} = \frac{m}{s}$$
			\subsubsection{Determine the transmission time of the packet $d_{trans}$ in terms of $L$ and $R$}
				$$d_{trans}=\frac{L}{R}$$
			\subsubsection{ Ignoring processing and queuing delays, obtain an expression for the end-to-end delay}
				$$d_{trans}+d_{prop}=\frac{m}{s}\cdot \frac{L}{R}$$
			\subsubsection{Suppose Host A begins to transmit the packet at time $t=0$. At time $t=d_{trans}$ where is the last of the packet}
				It would be in the link, since it would then have gathered the entire package
			\subsubsection{Suppose $d_{prop}$ is greater than $d_{trans}$. At time $t=d_{trans}$ where si the first bit of the packet.}
				It would be in the link, since the link have to gather the entire package before sending it off
			\subsubsection{Suppose $s=2.5\cdot 10^8$, $L=120$bits and $R=56$kbps. Find the distance $m$ so that $d_{prop}=d_{trans}$}
				\begin{align*}
					d_{trans}=\frac{L}{R}=\frac{120}{56000}=0.00214\\
					d_{prop}=\frac{m}{s}=\frac{m}{2.5\cdot 10^8}=d_{trans}\\
					m=d_{trans}\cdot 2.5\cdot 10^8=535.714
				\end{align*}
				Assuming that $s$ is in the unit meters/sec the distance would be 535.714 meters
		\subsection{Suppose N packets arrive simultaneously to a link at which no packets are currently being transmitted or queued. Each packet is of length L and the link has transmission rate R.}
			\subsubsection{What is the average queing delay for the $n$ packets}
				$$\frac{L\cdot P/2}{R}$$
				On average the packet would be in the middle of queue and therefore be half of $P$
			\subsubsection{Now suppose that N such packets arrive to the link every LN/R seconds. What is the average queuing delay of a packet?}
				Since the package arrival is faster than the package handling time, in the case of an infinite size buffer the aver delay would be infinite
		\subsection{Suppose two hosts A and B are speareated by 20,000 kilometers and are conencted by a direct link of R=2 Mbs. Suppose the propagation speed of the link is $s=2.5\cdot 10^8$ meters/sec}
			\subsubsection{Calculate the bandwidth-delay product $R\cdot d_{prop}$}
				$$R\cdot \frac{m}{s}=2\cdot 10^6b/s\cdot \frac{20,000,000m}{2.5\cdot 10^8m/s}=160Kb$$
			\subsubsection{Consider sending a file of 800,000 bits from Host A to Host B. Suppose the file is sent continuously as one large message. What is the maximum number of bits that will be in the link at any given time}
				160Kb as found in last exercise
			\subsubsection{Provide an interpretation of the bandwidth-delay product}
				Bandwidth-delay product is the number if bits which can exist in the link at one time every second.
			\subsubsection{What is the width of a bit in the link}
				$$\frac{20,000,000m}{160,000b}=125m$$
			\subsubsection{Derive a general expression for the width of a bit in terms of the prpagation speed $s$, the transmission rate $R$ and the length of the link $m$.}
				$$\frac{m}{R\cdot \frac{m}{s}}=\frac{s}{R}$$
		\subsection{Suppose there is a 10 Mbps microwave link between a geostationary satellite and its base station on Earth. Every minut the satellite takes a digital photo and sends it to the base station. Assume a propagation speed of $s=2.5\cdot 10^8$ meters/sec.}
			\subsubsection{What is the propagation delay of the link}
				$$\frac{m}{s}= \frac{35,786,000m}{2.5\cdot 10^8m/s}= 0.143s$$
			\subsubsection{What is the bandwidth delay product of the link}
				$$R\cdot \frac{m}{s}=10^7b\cdot \frac{35,786,000m}{2.5\cdot 10^8m/s}=1.43Mb$$
			\subsubsection{Let x denote the size of the photo. What is the minimum value of x for the microwave link the be contrinously transmitting}
				There are 86400 seconds in a day, therefore the minumum size of the photo would be:
				$$1.43Mb/s \cdot 86400s = 123.5Gb$$
		\subsection{Would it be faster to ship 300 terabytes over night than transfer with 1 Gbps}
			$$\frac{300,000Gb}{1Gb/s}=300,000s= 83.33 hours$$
			it would therefore be faster to ship the harddrive
	\section{Lecture 2}
		\subsection{Assume you request a webpage consisting of one document and five images. The document size is 1 kbyte, all images have the same size of 50 kbytes, the download rate i 1 Mbps, and the RTT is 100 ms. How long does it take to obtain the whole webpage under the following conditions? }
			\begin{itemize}
				\item Nonpersistent HTTP with serial connection - 
				\begin{align*}
					= 6 \cdot 2\cdot 100ms + (1kbyte + (5\cdot 50)kbytes)/1Mbps\\
					=1200ms + (8kb + 2000kb)/1000kbps\\
					=1200ms + 2008ms\\
					=3208ms
				\end{align*}
				\item Nonpersistent HTTP with two parrallel connections - 
				\begin{align*}
					= 6 \cdot 2\cdot 100ms / 2 + (1kbyte + (5\cdot 50)kbytes)/1Mbps\\
					=600ms + (8kb + 2000kb)/1000kbps\\
					=600ms + 2008ms\\
					=2608ms
				\end{align*}				
				\item Nonpersistent HTTP with six parrallel connections - 
				\begin{align*}
					= 6 \cdot 2\cdot 100ms / 6 + (1kbyte + (5\cdot 50)kbytes)/1Mbps\\
					=200ms + (8kb + 2000kb)/1000kbps\\
					=200ms + 2008ms\\
					=2208ms
				\end{align*}			
				\item Persistent HTTP with one connection -  
				\begin{align*}
					= 2\cdot 100ms + (1kbyte + (5\cdot 50)kbytes)/1Mbps\\
					=200ms + (8kb + 2000kb)/1000kbps\\
					=200ms + 2008ms\\
					=2208ms
				\end{align*}	
			\end{itemize}
		\subsection{Explain the mechanism used for signaling between the client and server to indicate that a persistent connection is being closed. Can the client, the server, or both signal the close of a connection?}
			A persistent connection will be interpreted as persistent until the general header field Connection is set to close by either client or server
		\subsection{What encryption services are provided by HTTP}
			HTTP does not support any encryption, at best the application itself can implement an encryption itself.
		\subsection{Can a client open three or more simultaneous connections with a given server}
			There is no limit to the amount of simultaneous connections but in case of consistent connection a larger amount would not benefit
		\subsection{Either a server or a client may close a transport connection between them if either one detects the connection has been idle for some time. Is it possible that one side starts closing a connection while the other side is transmitting data via this connection? Explain}
			This may happend, if the connection is slow and being sent just at the end of the idle timer.\\
			This scenario should also be counted for, such in case a connection is closed upon sending it should be recoverable.
		\subsection{We have seen that Internet TCP sockets treat the data being sent as a byte stream but UDP sockets recognize message boundaries. What are one advantage and one disadvantage of byte-oriented API versus having the API explicitly recognize and preserve application-defined message boundaries}
			The advantage to knowing the message boundery would be less control is needed.\\
			In the scenario of two commands being TIME and TIME-OF-DAY, in case of TCP when sending TIME-OF-DAY it may be seperated such TIME comes first and a response is sent wrongfully.\\
			But on the flipside if larger files are transfered, the TCP would simply gather every chunk of the file while UDP would have to gather it and ensure the right order of the cunks.
		\subsection{SMS, iMessage, and WhatsApp are all smartphone real-time messaging systems. After doing some research on the internet, for each of these systems write one paragraph about how the protocols they use. Then write a paragraph explaining how they differ.}
			\begin{itemize}
				\item SMS - Uses a protocol named SMPP, which is based upon TCP. The entity ESME connects to a provider SMSC and begin to send a SMPP with the message, a submit message is then sent to the SMSC, which follows by a success message back to the ESME.
				\item IMessage - A proprietary protocol which connects to apples servers with a TCL encryption. The protocol is based on XMPP which is a protocol for streaming XML elements over networks.
				\item WhatsApp - Also based on XMPP protocol and uses the signal protocol the encrypt its messages.
			\end{itemize} 
\end{document}
